\chapter{Abstract}
\label{chpr:abstract}

aaaaaaaaa