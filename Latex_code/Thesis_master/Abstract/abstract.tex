\chapter{Abstract}
\label{chpr:abstract}


Cryptoassets are becoming increasingly popular in the financial markets as a new investment instrument. This thesis aims to study their use in this sense from several points of view. After deepening on the state of art, this work analyzes cryptoassets properties such as returns and volatility in relation to the properties of common financial instruments such as equity, bonds, currencies and commodities represented by financial indices. Then the correlations are studied both within the world of cryptoassets and with all the other standard assets. What catches the eye is the fact that cryptoassets are uncorrelated to the market but strongly correlated to each other. This suggest to consider them as a new asset class.
Finally an analysis of the optimal allocation of a portfolio composed with and without cryptoassets is done with the Markowitz model.
The results show that it is useful in terms of diversification to include cryptoassets in a portfolio but having multiple cryptoassets or just bitcoins doesn't make much difference.
