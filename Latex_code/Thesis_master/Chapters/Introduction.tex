\chapter{Introduction}
\label{chpr:intro}

\bigskip



Bitcoin was presented in \citep{BTC2008} as a peer-to-peer protocol for electronic cash that would allow online payments to be sent directly from one party to another without going through a financial institution.
It started off in 2009 as system that was only used by a niche of people in online cryptology forums  to experiment with the transaction protocol. It took a few years for Bitcoin to gain public notoriety, while the price kept rising and having quick crashes. 

In particular, Bitcoin generated a lot of buzz in 2017, a year that registered the price increase from 998 \$ to 13,412 \$ in January 2018 with an all-time high of 19,666 \$ on the 17th of December. 
Like during any of history's gold rushes, many people joined the trend and tried to jump on the train  of quick and easy money and were let down when the price deflated to a level of six thousand dollars in 2018 and has lately stabilized around three thousands.

Setting aside the mere numerical value of the price, Bitcoin was the first protocol to solve the problem of double-spending without the need for a centralized party: bitcoins can be transferred but not duplicated, as they only exist as validated transaction in the distributed blockchain. 
These features allow Bitcoin to have a chance at becoming a global, instantaneous and free payment network that would make wealth transfer as easy as online data sharing.
In the same way that e-mail substituted post mail, Wikipedia and other knowledge-based website outdated paper encyclopaedias, music and film streaming services are becoming the new user friendly experience for the two industries, Bitcoin presents itself as a system to exchange wealth between users without the need for banks or other trusted third parties.

Furthermore, Bitcoin is the first digital currency to achieve scarcity in the digital realm: its monetary policy based on deterministic supply mimics the progressive scarcity of gold. It is this \textit{deterministic} supply that make Bitcoin a suitable long term investment, since it prevents arbitrary increase in its total available amount, unlike what happens with \textit{fiat} money.

\bigskip
For these reasons we believe that Bitcoin can be  \textit{digital gold} with an embedded secure network and its characteristics make it resemble more closely a crypto-asset rather than a crypto-currency.
Even though there are a number of supporters of this idea, for instance in \citep{DYHRBERG2016} it is shown that Bitcoin possesses the same hedging abilities of gold, there is yet no general consensus on the matter and different studies get to the opposite result: see for example \citep{KLEIN2018}.

Bitcoin has been called many names: a bubble, a Ponzi scheme, but also defined as sound money and store of value. We agree with the latter and believe that if Bitcoin is  true digital gold, then its value will express the huge potential that has been so far limited by scepticism and misunderstandings.

\bigskip

In the present work, we intend to first study the correlation that exists between Bitcoin and other types of standard assets, both by analysing the empirical correlation of the returns with its significance and by calibrating more sophisticated models such as \textit{jump diffusion} and \textit{stochastic volatility} models.
Secondly, we want to explore the diversification property of adding Bitcoin to a portfolio of asset, by computing the optimal allocation for different levels of risk and expected return in a Markowitz mean-variance framework and by optimizing on the CVaR as the portfolio risk measure.

\bigskip

The inclusion of Bitcoin in an investment portfolio is strongly suggested in \citep{hedge_safe_haven} where the authors arrive to the conclusion that the digital currency is indeed a great diversifier, while not performing as well as a hedge or safe haven. In \citep{andrianto} the diversification properties of Bitcoin are studied by inserting it into a portfolio composed of 8 assets including stocks, currencies and commodities and performing a Markowitz mean-variance analysis.
Finally in \citep{caveat_emptor} the authors study the performance of a portfolio that includes Bitcoin  adopting a CVaR framework: the result is that a small percentage of wealth should be invested in the digital asset.
In the last part of our thesis, we aim to update the results of the papers we just referenced by including the most recent price data, considering a total of 16 assets (15 standard assets plus Bitcoin) and performing a comparison between the two asset allocation frameworks.
Our final results match the outcomes of those articles.

\bigskip
\noindent
The present thesis has been written during the author's internship at the Digital Gold Institute(\href{https://www.dgi.io/}{https://www.dgi.io/}), a research and development center focused on teaching, consulting, and advising about scarcity in the digital domain (Bitcoin and crypto-assets) and the underlying blockchain technology.


\section{Thesis structure}
\bigskip
aaaaaaaaaaaaaaaaaaaa


\section{Dataset  Presentation}
\bigskip

